\chapter{Grammar}\label{appendix:grammar}
The grammar of SPL looks as follows:

\begin{lstlisting}
SPL       = Decl*
Decl      = VarDecl
          | FunDecl
VarDecl   = ('var' | Type) id  '=' Exp ';'
FunDecl   = id '(' [ FArgs ] ')' [ '::' FunType ] '{' VarDecl* Stmt* '}'
RetType   = Type
          | 'Void'
FunType   = [Annote '=>'] Type* '->' RetType
Annote    = Class id [',' Annote]
Class     = 'Eq'
          | 'Ord'
          | 'Show'
Type      = BasicType
          | '(' Type ',' Type ')'
          | '[' Type ']'
          | id
BasicType = 'Int'
          | 'Bool'
          | 'Char'
FArgs     = id  [ ',' FArgs ]
Stmt      = 'if' '(' Exp ')' '{' Stmt* '}' [ 'else' '{' Stmt* '}' ]
          | 'while' '(' Exp ')' '{' Stmt* '}'
          | id Field '=' Exp ';'
          | FunCall ';'
          | 'return' [ Exp ] ';'
Exp       = id Field
          | Exp Op2 Exp
          | Op1 Exp
          | int
          | char
          | 'False' | 'True'
          | '(' Exp ')'
          | FunCall
          | '[]'
          | '(' Exp ',' Exp ')'
Field     = [ Field ( '.hd' | '.tl' | '.fst' | '.snd' ) ]
FunCall   = id '(' [ ActArgs ] ')'
ActArgs   = Exp [ ',' ActArgs ]
Op2       = '+'  | '-' | '*' | '/'  | '%' | '==' | '<' | '>' | '<='
          | '>=' | '!=' | '&&' | '||' | ':'
Op1       = '!'  | '-'
int       = digit+
id        = alpha ( '_' | alphaNum)*
char      = '\'' ascii '\''
ascii     = // any ASCII character
digit     = '0' | '1' | '2' | '3' | '4' | '5' | '6' | '7' | '8' | '9'
alpha     = 'a' | 'b' | 'c' | 'd' | 'e' | 'f' | 'g' | 'h' | 'i' | 'j'
          | 'k' | 'l' | 'm' | 'n' | 'o' | 'p' | 'q' | 'r' | 's' | 't'
          | 'u' | 'v' | 'w' | 'x' | 'y' | 'z' | 'A' | 'B' | 'C' | 'D'
          | 'E' | 'F' | 'G' | 'H' | 'I' | 'J' | 'K' | 'L' | 'M' | 'N'
          | 'O' | 'P' | 'Q' | 'R' | 'S' | 'T' | 'U' | 'V' | 'W' | 'X'
          | 'Y' | 'Z'
alphaNum  = alpha | digit
\end{lstlisting}

Note that code may also contain comments, which either start with ``//'' and end with a newline character, or start with ``/*'' and end with the first encountered ``*/''. These can occur anywhere in the file.

To counter the ambiguity of the operators in the grammar above, we define the precedence and associativity of each operator in table \ref{table:precedence}.

\begin{table}[!ht]
\centering
\begin{tabular}{|l|l|l|l|}
\hline
\textbf{Operator} & \textbf{Precedence} & \textbf{Associativity} & \textbf{Prefix/infix} \\ \hline
@-@ & 17 & Right & Prefix \\ \hline
@!@ & 7 & Right & Prefix \\ \hline
@*@ & 15 & Left & Infix \\ \hline
@/@ & 15 & Left & Infix \\ \hline
@%@ & 15 & Left & Infix \\ \hline
@+@ & 13 & Left & Infix \\ \hline
@-@ & 13 & Left & Infix \\ \hline
@<@ & 11 & Left & Infix \\ \hline
@>@ & 11 & Left & Infix \\ \hline
@<=@ & 11 & Left & Infix \\ \hline
@>=@ & 11 & Left & Infix \\ \hline
@==@ & 9 & Left & Infix \\ \hline
@!=@ & 9 & Left & Infix \\ \hline
@&&@ & 5 & Right & Infix \\ \hline
@||@ & 3 & Right & Infix \\ \hline
@:@ & 1 & Right & Infix \\ \hline
% @.fst@ & 19 & Left & Postfix \\ \hline
% @.snd@ & 19 & Left & Postfix \\ \hline
% @.hd@ & 19 & Left & Postfix \\ \hline
% @.tl@ & 19 & Left & Postfix \\ \hline
\end{tabular}
\caption{The precedence and associativity of each operator.}
\label{table:precedence}
\end{table}

% \subsection{Changes}
% \begin{itemize}
%     \item SPL now contains many declarations (can be 0)
%     \item FunDecl now contains many statements (can be 0)
%     \item FunType now contains many types (rewritten)
%     \item Fargs structure now same as ActArgs (the implementation abstracts both away)
%     \item Removed minus before int (it's already an operator)
%     \item Added char, digit, alpha, and alphaNum (note that these may not be separated with spaces)
% \end{itemize}
